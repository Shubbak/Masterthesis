\chapter{Theoretische Grundlagen} \label{sec:theorie}

\section{Franck-Condon Principle}

\begin{figure}[tbhp]
    \centering
    
\tikzmath{
    function morse(\x, \De, \a, \re, \v1, \r1) {
        return \v1 + \De * (1 - exp(-\a * (\x - (\re + \r1))))^2;
    };
    \DX = 4.52;  % = equil. En.
    \DB = 4.1;  % = equil. En.
    \rX = 0.741; % = equil. distance
    \rB = 0.732; % = equil. distance
    \a = 1.942; % spring const.
    \aB = 2.014; % spring const.
    \x1 = 0.2; % domain
    \x2 = 3;   % domain
    \r1 = 0.1; % translation of equil. distance
    \v1 = 6;    % translation of equil. energy
}


\tikzmath{\x0 = 0.7; \x3 = 0.9; \x4 = 1.1;} % fixed points

\begin{tikzpicture}
    \begin{axis}[
        %title={Morse Potential for $\mathrm{H_2}$ in the $X$ and $B$-States},
        xlabel={$\sfrac{r}{\si{\angstrom}}$},
        ylabel={$\sfrac{V(r)}{\si{\eV}}$},
        ylabel style={rotate=-90},
        domain=0.5:5, samples=100
        ]
        \addplot[name path = X, thick, samples=1000, domain=\x1:\x2]
            {morse(x, \DX, \a, \rX, 0, 0)}
            node[below,  pos=0.95]{$X$};
        \addplot[name path = B, thick, red, samples=1000, domain=\x1 + \r1:\x2]
            {morse(x, \DB, \aB, \rB, \v1,0)}
            node[below, pos=0.95] {$B$};
        % \draw[->, thick] let \p1 = (X), \p2 = (B) in (\p1) -- (\p2) ;

        % \draw[->, dashed] 
            % (axis cs: \x0 , {morse(\x0, \DX, \a, \rX, \v1, \r1)}) --
            % (axis cs: \x0 , {morse(\x0, \DX, \a, \rX, 0, 0)});
        % \draw[->, dotted] 
            % (axis cs: \x3 , {morse(\x3, \DX, \a, \rX, \v1, \r1)}) --
            % (axis cs: \x3, {morse(\x3, \DX, \a, \rX, 0, 0)});
        % \draw[->, dashdotted] 
            % (axis cs: \x4 , {morse(\x4, \DX, \a, \rX, \v1, \r1)}) --
            % (axis cs: \x4 , {morse(\x4, \DX, \a, \rX, 0, 0)});
    \end{axis}
\end{tikzpicture}


    \caption{Die potentielle Energie des Grundzustandes $X$ und des ersten angeregten Zustandes $B$ des Wasserstoffmoleküls.}
    \label{fig:potentielle_Energie_X_und_B}
\end{figure}

\section{Computational Methods}

\subsection{Potential Energy Curves}

From literature, we acquire the adiabatic potential energy of the ground state $X$ and the excited states  $B, C, \ldots$ of the hydrogen molecule.
Those energies are given in tables of the form:
\begin{table}[H]
    \centering
    \begin{tabular}{c|c|c|c|c}
        $R$  & $E$  & $\dv{E}{r}$ & A & G  \\
        \hline
       0.50000& 0.1421519042E+00&-0.3347937156E+01&-0.1696142205E+00&-0.7926117348E+00\\
       0.55000&-0.7377199290E-02&-0.2666742446E+01&-0.1747251081E+00&-0.7641982319E+00\\
       0.60000&-0.1273149179E+00&-0.2154021077E+01&-0.1786011423E+00&-0.7374615090E+00\\
       0.65000&-0.2247458012E+00&-0.1759832547E+01&-0.1815016671E+00&-0.7123468247E+00\\
       0.70000&-0.3047210135E+00&-0.1451349142E+01&-0.1836347520E+00&-0.6887785242E+00\\
    \end{tabular}
    \caption{Potential energies of the hydrogen molecule in the ground state and the first excited state}
    \label{tab:potential_energies}
\end{table}
Where $A$ is the nuclear Laplacian %TODO
and  $G$ is  $\frac{\laplacian_1 + \grad_1 \cdot \grad_2}{2}$.

The program \texttt{basic\_lit\_potentials.py} reads the $R$ and  $E$ data from those tables and, after converting the units to \si{\angstrom} and \si{\eV}, it writes the data to a \texttt{.h5}-file in the format:

\begin{verbatim}
    |--H2_literatre.h5
    |   |-- X
    |   |   |-- potential
    |   |-- B
    |   |   |-- potential
    etc.
\end{verbatim}
and each \texttt{potential} contains a ($n$,2) array with the first coordinate being the distance $R$ and the second coordinate being the potential energy $E$.

\subsection{Transition Moments}
With tables from literature, in which the transition moments with respect to the internuclear distance $R$ are given from the excited states to the ground state, that have the form:
\begin{table}[htpb]
    \centering
    \caption{caption}
    \label{tab:label}
    \begin{tabular}{c|c|c|c|c}
        $R$    &     $B - X$    &   $C - X$    &   $D - X$  &   $V - X$    \\ 
        \hline
        0.50 & 0.51411 & 0.24787 & 0.00010 & 0.15470\\
        0.55 & 0.52718 & 0.25335 & 0.00012 & 0.15796 \\
        0.60 & 0.54038 & 0.25888 & 0.00015 & 0.16125 \\
        0.65 & 0.55367 & 0.26438 & 0.00018 & 0.16451 \\
    \end{tabular}
\end{table}
The program \texttt{basis\_lit\_moments.py} simply reads the data and inserts it into the \texttt{.h5}-file, so that the final format becomes:
\begin{verbatim}
    |--H2_literatre.h5
    |   |-- X
    |   |   |-- potential
    |   |   |-- transition_X
    |   |-- B
    |   |   |-- potential
    |   |   |-- transition_X
    etc.
\end{verbatim}

where \texttt{transition\_X} contains a ($n$,2) array with the first coordinate being the distance $R$ and the second coordinate being the transition moment $T$.

\subsection{Computing the wave functions}
The program \texttt{schroedinger\_mp.py} using the \texttt{.h5}-file generated by the previos two programs to solve the Schrödinger equation and compute the wave functions of the hydrogen molecule.

\begin{verbatim}
    |--H2_calculated.h5
    |   |-- X
    |   |   |-- levels
    |   |   |-- potential
    |   |   |-- wavefunctions
    |   |-- B
    |   |   |-- levels
    |   |   |-- potential
    |   |   |-- wavefunctions
    etc.
\end{verbatim}

where \texttt{levels} are the energies at a given pair of $v,J$, \texttt{potential} is the interpolated energy curve,  \texttt{wavefunctions} are the vibrational wave functions with the given grid at a specified  $J$






